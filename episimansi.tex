\begin{center}
    \LARGE \textbf{Μαθηματικά Γ’ Λυκείου}\\[1ex]
    \large Θεωρία \& Συνοπτική Μεθοδολογία\\
    \normalsize Μόνο για Επανάληψη.
\end{center}

\vspace{1em}

\textbf{Επισήμανση στην θεωρία:}

\vspace{0.5em}
\begin{itemize}[leftmargin=1em]

    \item \textbf{Σταθερές Μεταβλητές:}
    \begin{itemize}
        \item $e = 2.71\ldots$
        \item $\pi = 3.14\ldots$
    \end{itemize}

    \item \textbf{Ο φυσικός λογάριθμος είναι ο λογάριθμος με βάση ``e'':}
    \[
    \ln \theta = \log_e \theta
    \]

    \item \textbf{Βασική ιδιότητα του λογάριθμου:}
    \[
    \log_\alpha \theta = x \iff \alpha^x = \theta
    \]

    \item \textbf{Σύνολα Αριθμών:}
    \[
    \mathbb{N} \subseteq \mathbb{Z} \subseteq \mathbb{Q} \subseteq \mathbb{R}
    \]

    \begin{itemize}
        \item $\mathbb{N}$ : Φυσικοί αριθμοί.
        \begin{itemize}
            \item Δηλαδή $\{1, 2, 3, 4, 5, \ldots\}$
        \end{itemize}

        \item $\mathbb{Z}$ : Ακέραιοι Αριθμοί.
        \begin{itemize}
            \item Δηλαδή $\{\ldots, -3, -2, -1, 0, 1, 2, 3, 4, 5, \ldots\}$
        \end{itemize}

        \item $\mathbb{Q}$ : Ρητοί αριθμοί
        \begin{itemize}
            \item Δηλαδή όλοι οι αριθμοί που μπορούν να αναπαρασταθούν με ένα κλάσμα ακέ\-ραιων αριθμών.
        \end{itemize}

        \[
        \mathbb{Q} = \left\{ \frac{\alpha}{\beta} \;\middle|\; \alpha \in \mathbb{Z}, \beta \in \mathbb{Z}^* \right\}
        \]

        \item $\mathbb{R}$ : Πραγματικοί αριθμοί
        \begin{itemize}
            \item Δηλαδή $\mathbb{R} = \mathbb{Q}\cup$Άρρητοι αριθμοι
        \end{itemize}

        \item \textbf{Αξίζει να σημειωθεί} πως ο αστερίσκος στα σύνολα των αριθμών συμβολίζει το παρα\-κάτω:
        \begin{itemize}
            \item $\mathbb{N}^* = \mathbb{N} - \left\{ 0 \right\}$
            \item $\mathbb{Z}^* = \mathbb{Z} - \left\{ 0 \right\}$
            \item $\mathbb{Q}^* = \mathbb{Q} - \left\{ 0 \right\}$
            \item $\mathbb{R}^* = \mathbb{R} - \left\{ 0 \right\}$
        \end{itemize}


    \end{itemize}

    \item \textbf{Διαστήματα:}
    \begin{itemize}
        \item $x \in ( \alpha, \beta) \iff \alpha < x < \beta$
        \item $x \in [ \alpha, \beta) \iff \alpha \leq x < \beta$
        \item $x \in ( \alpha, \beta] \iff \alpha < x \leq \beta$
        \item $x \in [ \alpha, \beta] \iff \alpha \leq x \leq \beta$
    \end{itemize}

\newpage

    \item \textbf{Ιδιότητες Ανισοτήτων:}
    \begin{itemize}
        \item Για $\alpha < \beta$ με $\alpha, \beta \in \mathbb{R}$ και έστω $\gamma \in \mathbb{R}$, ισχύει:
        \[
        \alpha < \beta \iff \alpha + \gamma < \beta + \gamma
        \]
        \[
        \alpha < \beta \iff \alpha - \gamma < \beta - \gamma
        \]

        \item Για $\alpha < \beta$ με $\alpha, \beta \in \mathbb{R}$, έχουμε:
        \begin{itemize}
            \item Για κάθε $\gamma > 0$:
            \[
            \alpha < \beta \iff \alpha \cdot \gamma < \beta \cdot \gamma
            \]
            \item Για κάθε $\gamma < 0$:
            \[
            \alpha < \beta \iff \alpha \cdot \gamma > \beta \cdot \gamma
            \]
        \end{itemize}

        \item Επιτρέπεται η πρόσθεση κατά μέλη (\textbf{ΜΟΝΟ} για την \textbf{ίδια} ανισοτική φορά).
        Η ``αφαίρεση'' μπορεί να γίνει μόνο με τα παρακάτω βήματα. Έστω $\alpha < \beta$ και $\gamma < \delta$:

        \begin{enumerate}
            \item Πολλαπλασιάζω με το $-1$:
            \[
            \alpha < \beta \iff -\alpha > -\beta
            \]
            \item Προσθέτω κατά μέλη:
            \[
            -\beta < -\alpha, \quad \gamma < \delta \implies \gamma - \beta < \delta - \alpha
            \]
        \end{enumerate}

        \item Αν $\alpha > 0$ και $\beta > 0$, τότε:
        \[
        \alpha < \beta \iff \alpha^\nu < \beta^\nu \quad \text{όπου } \nu \in \mathbb{N}^*
        \]

        \item Αν όμως $\alpha, \beta \in \mathbb{R}^*$, τότε:
        \[
        \alpha < \beta \iff \alpha^{2\nu + 1} < \beta^{2\nu + 1} \text{ όπου } \nu \in \mathbb{N}^*
        \]
        Ουσιαστικά, το $2\nu + 1 =$ περιττός αριθμός. Άρα θα μπορούμε να πούμε $\nu \in \{1,3,5,7,9,\ldots\}$ %ignore
        \item Άν $\alpha, \beta \gtrless 0$ (δηλαδή $\alpha,\beta$ ομόσημα):
        \[
        \alpha \gtrless \beta \iff \frac{1}{\alpha} \lessgtr \frac{1}{\beta}
        \]


    \end{itemize}

\end{itemize}
