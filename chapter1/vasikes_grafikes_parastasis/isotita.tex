{\large \textbf{- Ισότητα Συναρτήσεων}}
\vspace{1em}

\begin{tcolorbox}[colback=blue!10!white, colframe=blue!50!black, title=Ορισμός Ισότητας Συνάρτησης (σελ.23)]
Δύο συναρτήσεις $f: A \to \mathbb{R}$ και $g: B \to \mathbb{R}$ λέγονται \textbf{ίσες} όταν ισχύουν τα εξής:
\begin{enumerate}
  \item Τα πεδία ορισμού τους είναι ίσα: $A = B$
  \item και για κάθε $x \in A$ έχουμε $f(x) = g(x)$
\end{enumerate}
\end{tcolorbox}

\vspace{0.5em}

\textbf{Παρατηρήσεις:}
\begin{itemize}
  \item Αν τα πεδία ορισμού διαφέρουν, τότε οι συναρτήσεις \textbf{δεν} θεωρούνται ίσες, ακόμη κι αν έχουν ίδιους τύπους τύπου (π.χ. $f(x)=x^2$ με $A=\mathbb{R}$ και $g(x)=x^2$ με $B=[0,+\infty)$ δεν είναι ίσες).
  \item Για να αποδείξουμε ότι δύο συναρτήσεις είναι ίσες, αρκεί να δείξουμε ότι έχουν κοινό πεδίο ορισμού και ίδιες τιμές για κάθε $x$ (δηλάδή αρκεί να δείξουμε ότι έχουν και τον ίδιο τύπο).
\end{itemize}
