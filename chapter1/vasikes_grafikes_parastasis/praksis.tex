{\large \textbf{- Πράξεις Συναρτήσεων}}
\vspace{1em}

Έστω δύο συναρτήσεις $f: A \to \mathbb{R}$ και $g: B \to \mathbb{R}$.
Ορίζουμε τις παρακάτω πράξεις για $x \in A \cap B$:

\begin{itemize}
  \item \textbf{Άθροισμα:} $(f+g)(x) = f(x) + g(x)$
  \item \textbf{Διαφορά:} $(f-g)(x) = f(x) - g(x)$
  \item \textbf{Γινόμενο:} $(f \cdot g)(x) = f(x) \cdot g(x)$
  \item \textbf{Και για την διαίρεση (λόγος/πηλίκο):}
  \[
   \left(\frac{f}{g}\right)(x) = \frac{f(x)}{g(x)}, \quad \text{για } x \in A \cap B \text{ και } g(x) \neq 0
  \]
\end{itemize}

\vspace{0.5em}

\textbf{Παρατηρήσεις:}
\begin{itemize}
  \item Το πεδίο ορισμού κάθε πράξης είναι τομή των πεδίων ορισμού $A \cap B$, με επιπλέον περιο\-ρισμούς όπου χρειάζεται (π.χ. στο πηλίκο: $g(x)\neq 0$).
  \item Οι πράξεις συναρτήσεων κληρονομούν ιδιότητες από τις πράξεις αριθμών:
    \begin{itemize}
      \item $(f+g)(x) = (g+f)(x)$ (αντιμεταθετική ιδιότητα),
      \item $(f \cdot g)(x) = (g \cdot f)(x)$,
      \item $(f+(g+h))(x) = ((f+g)+h)(x)$ (προσεταιριστική ιδιότητα),
      \item $(f \cdot (g+h))(x) = f \cdot g(x) + f \cdot h(x)$ (διανεμητική ιδιότητα).
    \end{itemize}
\end{itemize}

