{\large \textbf{- Σύνθεση Συναρτήσεων}}

\begin{tcolorbox}[colback=blue!10!white, colframe=blue!50!black, title=Ορισμός Σύνθεσης Συναρτήσεων]
Αν $f$ και $g$ είναι δυο συναρτήσεις με πεδίο ορισμού $A$ και $B$ αντιστοίχως, τότε ονομάζουμε σύνθεση της $f$ με την $g$, και την συμβολίζουμε με $g \circ f$ την συνάρτηση με τύπο $(g \circ f)(x) = g(f(x))$
\end{tcolorbox}


Λίγο πιο αναλυτικά ο ορισμός. Έστω δύο συναρτήσεις
\[
f: A \to \mathbb{R}, \quad g: B \to \mathbb{R}.
\]

Η \textbf{σύνθεση} της $f$ με την $g$ είναι η συνάρτηση
\[
g \circ f : D \to \mathbb{R}
\]
που ορίζεται ως εξής:
\[
(g) \circ f)(x) = g(f(x)), \quad \text{για κάθε } x \in D,
\]
όπου
\[
D = \{ x \in A \;|\; f(x) \in B \}.
\]
Για την διευκόλυνση μας, ο παραπάνω τύπος για τον πεδίο ορισμού για την σύνθεση της $f$ με την $g$ μπορεί να βγει πιο εύκολα όταν αναλύουμαι πιο πρέπει να είναι το πεδίο ορισμού από τον δεύτερο συμβολισμό της σύνθεσης: $g(f(x))$. Το $x$ φυσικά πρέπει $x \in A$ και το $f(x)$ φυσικά πρέπει $f(x) \in B$.

\vspace{1em}

\textbf{Παρατηρήσεις:}
\begin{itemize}
  \item Η $f \circ g$ διαβάζεται «$f$ σύνθεση $g$» και σημαίνει ότι πρώτα εφαρμόζεται η $g$ και στη συνέχεια η $f$.
  \item Η σύνθεση \textbf{δεν είναι αντιμεταθετική}, δηλαδή γενικά:
  \[
  f \circ g \neq g \circ f.
  \]
  \item Η σύνθεση είναι \textbf{προσεταιριστική}, δηλαδή:
  \[
  f \circ (g \circ h) = (f \circ g) \circ h.
  \]
\end{itemize}

{\large \textbf{Μεθοδολογία για Σύνθεση και Αποσύνθεση Συναρτήσεων}}
\vspace{1em}

\textbf{1. Σύνθεση Συναρτήσεων}

Έστω δύο συναρτήσεις
\[
f: A \to \mathbb{R}, \quad g: B \to \mathbb{R}.
\]

Η σύνθεση της $f$ με την $g$, που στο σχολικό βιβλίο συμβολίζεται $g \circ f$, ορίζεται ως
\[
(g \circ f)(x) = g(f(x)), \quad x \in D,
\]
όπου
\[
D = \{ x \in A \mid f(x) \in B \}.
\]

\vspace{0.5em}

\textbf{Βήματα για να βρούμε τη σύνθεση $g \circ f$:}
\begin{enumerate}
  \item Βεβαιωνόμαστε ότι $f(x) \in B$, δηλαδή το $x$ ανήκει στο πεδίο $D$ της σύνθεσης.
  \item Υπολογίζουμε $f(x)$.
  \item Αντικαθιστούμε το $f(x)$ στη συνάρτηση $g$.
  \item Ορίζουμε το νέο πεδίο ορισμού $D$ όπου ισχύει το βήμα 1.
\end{enumerate}

\vspace{1em}

\textbf{Παράδειγμα Σύνθεσης:}
\[
f(x) = x^2, \quad g(x) = \sqrt{x+1} \quad \Rightarrow \quad (g \circ f)(x) = \sqrt{x^2+1}, \quad D = \mathbb{R}.
\]

\vspace{1em}

\textbf{2. Αποσύνθεση Συναρτήσεων}

Η αποσύνθεση μιας συνάρτησης $h(x)$ σημαίνει να τη γράψουμε ως σύνθεση δύο συναρτήσεων $f$ και $g$, δηλαδή
\[
h(x) = (g \circ f)(x) = g(f(x)).
\]

\vspace{0.5em}

\textbf{Βήματα για αποσύνθεση:}
\begin{enumerate}
  \item Εντοπίζουμε ένα "εσωτερικό" μέρος της συνάρτησης που μπορεί να οριστεί ως $f(x)$.
  \item Ορίζουμε τη συνάρτηση $g(x)$ ώστε να εφαρμόζει στο αποτέλεσμα της $f(x)$ και να δίνει το $h(x)$.
  \item Ελέγχουμε ότι το πεδίο ορισμού της σύνθεσης $g \circ f$ είναι σωστό.
\end{enumerate}

\vspace{1em}

\textbf{Παράδειγμα Αποσύνθεσης:}
\[
h(x) = \sqrt{3x+1}.
\]

\textbf{Λύση:}
\[
f(x) = 3x+1, \quad g(x) = \sqrt{x}, \quad h(x) = g(f(x)).
\]

\vspace{0.5em}

\textbf{Σημείωση:} Η αποσύνθεση δεν είναι μοναδική. Για παράδειγμα, μπορούμε επίσης να πάρουμε:
\[
f(x) = x, \quad g(x) = \sqrt{3x+1}, \quad h(x) = g(f(x)).
\]
