\documentclass[a4paper,12pt]{article}
\usepackage{fontspec}
\usepackage{polyglossia}
\setmainlanguage{greek}
\setotherlanguage{english}
\setmainfont{LiberationSans} % or another font that supports Greek
\usepackage{amsmath, amssymb}
\usepackage[most]{tcolorbox}
\usepackage{xcolor}
\usepackage{enumitem}
\usepackage[a4paper, margin=1.8cm]{geometry}

\usepackage{tikz}
\usetikzlibrary{decorations.pathreplacing}


\setlist[itemize]{left=1.5em, label=\textbullet}
\setlength{\parskip}{0pt}
\setlength{\parindent}{0pt}



\begin{document}

\begin{center}
    \LARGE \textbf{Μαθηματικά Γ’ Λυκείου}\\[1ex]
    \large Θεωρία \& Συνοπτική Μεθοδολογία\\
    \normalsize Μόνο για Επανάληψη.
\end{center}

\vspace{1em}

\textbf{Επισήμανση στην θεωρία:}

\vspace{0.5em}
\begin{itemize}[leftmargin=1em]

    \item \textbf{Σταθερές Μεταβλητές:}
    \begin{itemize}
        \item $e = 2.71\ldots$
        \item $\pi = 3.14\ldots$
    \end{itemize}

    \item \textbf{Ο φυσικός λογάριθμος είναι ο λογάριθμος με βάση ``e'':}
    \[
    \ln \theta = \log_e \theta
    \]

    \item \textbf{Βασική ιδιότητα του λογάριθμου:}
    \[
    \log_\alpha \theta = x \iff \alpha^x = \theta
    \]

    \item \textbf{Σύνολα Αριθμών:}
    \[
    \mathbb{N} \subseteq \mathbb{Z} \subseteq \mathbb{Q} \subseteq \mathbb{R}
    \]

    \begin{itemize}
        \item $\mathbb{N}$ : Φυσικοί αριθμοί.
        \begin{itemize}
            \item Δηλαδή $\{1, 2, 3, 4, 5, \ldots\}$
        \end{itemize}

        \item $\mathbb{Z}$ : Ακέραιοι Αριθμοί.
        \begin{itemize}
            \item Δηλαδή $\{\ldots, -3, -2, -1, 0, 1, 2, 3, 4, 5, \ldots\}$
        \end{itemize}

        \item $\mathbb{Q}$ : Ρητοί αριθμοί
        \begin{itemize}
            \item Δηλαδή όλοι οι αριθμοί που μπορούν να αναπαρασταθούν με ένα κλάσμα ακέ\-ραιων αριθμών.
        \end{itemize}

        \[
        \mathbb{Q} = \left\{ \frac{\alpha}{\beta} \;\middle|\; \alpha \in \mathbb{Z}, \beta \in \mathbb{Z}^* \right\}
        \]

        \item $\mathbb{R}$ : Πραγματικοί αριθμοί
        \begin{itemize}
            \item Δηλαδή $\mathbb{R} = \mathbb{Q}\cup$Άρρητοι αριθμοι
        \end{itemize}

        \item \textbf{Αξίζει να σημειωθεί} πως ο αστερίσκος στα σύνολα των αριθμών συμβολίζει το παρα\-κάτω:
        \begin{itemize}
            \item $\mathbb{N}^* = \mathbb{N} - \left\{ 0 \right\}$
            \item $\mathbb{Z}^* = \mathbb{Z} - \left\{ 0 \right\}$
            \item $\mathbb{Q}^* = \mathbb{Q} - \left\{ 0 \right\}$
            \item $\mathbb{R}^* = \mathbb{R} - \left\{ 0 \right\}$
        \end{itemize}


    \end{itemize}

    \item \textbf{Διαστήματα:}
    \begin{itemize}
        \item $x \in ( \alpha, \beta) \iff \alpha < x < \beta$
        \item $x \in [ \alpha, \beta) \iff \alpha \leq x < \beta$
        \item $x \in ( \alpha, \beta] \iff \alpha < x \leq \beta$
        \item $x \in [ \alpha, \beta] \iff \alpha \leq x \leq \beta$
    \end{itemize}

\newpage

    \item \textbf{Ιδιότητες Ανισοτήτων:}
    \begin{itemize}
        \item Για $\alpha < \beta$ με $\alpha, \beta \in \mathbb{R}$ και έστω $\gamma \in \mathbb{R}$, ισχύει:
        \[
        \alpha < \beta \iff \alpha + \gamma < \beta + \gamma
        \]
        \[
        \alpha < \beta \iff \alpha - \gamma < \beta - \gamma
        \]

        \item Για $\alpha < \beta$ με $\alpha, \beta \in \mathbb{R}$, έχουμε:
        \begin{itemize}
            \item Για κάθε $\gamma > 0$:
            \[
            \alpha < \beta \iff \alpha \cdot \gamma < \beta \cdot \gamma
            \]
            \item Για κάθε $\gamma < 0$:
            \[
            \alpha < \beta \iff \alpha \cdot \gamma > \beta \cdot \gamma
            \]
        \end{itemize}

        \item Επιτρέπεται η κατά μέλη πρόσθεση (\textbf{ΜΟΝΟ} για την \textbf{ίδια} ανισοτική φορά).
        Η ``αφαίρεση'' μπορεί να γίνει μόνο με τα παρακάτω βήματα. Έστω $\alpha < \beta$ και $\gamma < \delta$:

        \begin{enumerate}
            \item Πολλαπλασιάζω με το $-1$:
            \[
            \alpha < \beta \iff -\alpha > -\beta
            \]
            \item Προσθέτω κατά μέλη:
            \[
            -\beta < -\alpha, \quad \gamma < \delta \implies \gamma - \beta < \delta - \alpha
            \]
        \end{enumerate}

        \item Αν $\alpha > 0$ και $\beta > 0$, τότε:
        \[
        \alpha < \beta \iff \alpha^\nu < \beta^\nu \quad \text{όπου } \nu \in \mathbb{N}^*
        \]

        \item Αν όμως $\alpha, \beta \in \mathbb{R}^*$, τότε:
        \[
        \alpha < \beta \iff \alpha^{2\nu + 1} < \beta^{2\nu + 1} \text{ όπου } \nu \in \mathbb{N}^*
        \]
        Ουσιαστικά, το $2\nu + 1 =$ περιττός αριθμός. Άρα θα μπορούμε να πούμε $\nu \in \{1,3,5,7,9,\ldots\}$ %ignore
        \item Άν $\alpha, \beta \gtrless 0$ (δηλαδή $\alpha,\beta$ ομόσημα):
        \[
        \alpha \gtrless \beta \iff \frac{1}{\alpha} \lessgtr \frac{1}{\beta}
        \]


    \end{itemize}

\end{itemize}

\newpage

\begin{center}
    \LARGE \textbf{Ενότητα 1}\\[1ex]
    \Large \textbf{Κεφάλαιο 1.2}\\
\end{center}

\begin{tcolorbox}[colback=blue!10!white, colframe=blue!50!black, title=Ορισμός Συνάρτησης]
Έστω $A$ ένα υποσύνολο του $\mathbb{R}$. Ονομάζουμε πραγματική συνάρτηση με πεδίο ορισμου το $A$ μια διαδικασία (κανόνα) $f$ με την οποία κάθε στοιχείο $x\in A$ αντιστοιχίζεται σε έναν μόνο αριθμό $y$. Το $y$ ονομάζεται τιμή της $f$ στο $x$ και συμβολίζεται με $f(x)$.
\end{tcolorbox}

Συμβολικά τον παραπάνω ορισμό τον γράφουμε:
\begin{align*}
f \colon A &\to \mathbb{R}\\
x &\xrightarrow{f} y = f(x)\\
\end{align*}

\begin{itemize}
    \item $x$ : ανεξάρτητη μεταβλητή\\
    $y$ : εξαρτημένη μεταβλητή

    \item To πεδίο ορισμού της συνάρτησης ορίζεται και ως $A = D_{f}$


    \item Το σύνολο που έχει για στοιχεία του τις τιμές της $f$ σε όλα τα $x\in A$, λέγεται σύνολο τιμών της $f$ και συμβολίζεται με $f(A)$ (ή $f(D_{f})$). Είναι δηλαδή:
    \[
        f(A) = \left\{ y \mid y = f(x) \text{ για κάποιο } x \in A \right\}
    \]

    (Ο λόγος που λέει για κάποιο και όχι για κάθε είναι γιατί τότε θα εννοούσε ότι το ίδιο $y$ είναι αποτέλεσμα της $f$ για όλα τα $x$ στο $A$. Δηλαδή πως η $f$ είναι σταθερή ($f(x) = y$ για κάθε $x \in A$). Αυτό προφανώς δεν ισχύει για τις περισσότερες συναρτήσεις.)
\end{itemize}

\begin{center}
 \begin{tikzpicture}[scale=1]
  % Draw ellipses
  \draw[thick] (-3.5,0) ellipse (1.5cm and 2.5cm);
  \draw[thick] (0,0) ellipse (1.5cm and 2.5cm);
  \draw[thick] (5,0) ellipse (1.5cm and 2.5cm);
  \draw[thick] (8.5,0) ellipse (1.5cm and 2.5cm);

  \tikzset{dot/.style={circle, fill=black, inner sep=1.5pt}}


  % Nodes in the left ellipse
  \node[dot] (a1) at (-3,1.5) {};
  \node[dot] (a2) at (-4.5,1) {};
  \node[dot] (a3) at (-4,-0.5) {};
  \node[dot] (a4) at (-3.5,-1.5) {};

  % Nodes in the right ellipse
  \node[dot] (b1) at (0,1.8) {};
  \node[dot] (b2) at (0,0) {};
  \node[dot] (b3) at (0,-1) {};
  \node[dot] (b4) at (0.5,-1.5) {};

  % Nodes for the second pair left ellipse
  \node[dot] (c1) at (4.5,1.5) {};
  \node[dot] (c2) at (5.5,1) {};
  \node[dot] (c3) at (5,0) {};
  \node[dot] (c4) at (5.5,-0.5) {};
  \node[dot] (c5) at (4.5,-1.5) {};

  % Nodes for the second pair right ellipse
  \node[dot] (d1) at (8.8,1.8) {};
  \node[dot] (d2) at (9,0.7) {};
  \node[dot] (d3) at (9,-0.5) {};
  \node[dot] (d4) at (8.25,-1.8) {};

  % Curved arrows connecting dots (function mapping)
  \draw[->, thick, bend left=20] (a1) to (b2);
  \draw[->, thick, bend right=20] (a2) to (b1);
  \draw[->, thick, bend right=20] (a2) to (b2);
  \draw[->, thick, bend left=10] (a3) to (b3);
  \draw[->, thick, bend right=15] (a4) to (b4);

  \draw[->, thick, bend left=20] (c1) to (d2);
  \draw[->, thick, bend right=20] (c2) to (d1);
  \draw[->, thick, bend left=20] (c3) to (d3);
  \draw[->, thick, bend left=10] (c4) to (d4);
  \draw[->, thick, bend right=15] (c5) to (d3);

  % Labels
  \node[black, font=\bfseries] at (-3.5,3) {\Large{$x$}};
  \node[black, font=\bfseries] at (0,3) {\Large{$y$}};

  \node[black, font=\bfseries] at (5,3) {\Large{$x$}};
  \node[black, font=\bfseries] at (8.5,3) {\Large{$y$}};

  % More Labels

  \node[black, font=\bfseries] at (-1.7,-3.5) {Δεν είναι συνάρτηση.};
  \node[black, font=\bfseries] at (6.7,-3.5) {Είναι συνάρτηση.};


\end{tikzpicture}
\end{center}

\newpage




\begin{center}
 % You can also try \huge or \LARGE
\renewcommand{\arraystretch}{1.5} % Increases row height
\begin{tabular}{|l|p{10cm}|}
\hline
\textbf{Συνάρτηση} & \textbf{Περιορισμός} \\
\hline
$f(x) = \frac{P(x)}{Q(x)}$   & $Q(x) \neq 0$ \\
\hline
$f(x) = \sqrt[\nu]{P(x)} , \nu \in \mathbb{N}^* - {1}$    & $P(x) \ge 0$ \\
\hline
$f(x) = \ln(P(x))$    & $P(x) > 0$ \\
\hline
$f(x) = \varepsilon\varphi(P(x))$        & $P(x) \ne \kappa\pi + \frac{\pi}{2}, \kappa \in \mathbb{Z}$ \\
\hline
$f(x) = \sigma\varphi(P(x))$     & $P(x) \ne \kappa\pi, \kappa \in \mathbb{Z}$ \\
\hline
$f(x) = (P(x))^{Q(x)}$     & $P(x) > 0$ ή ($P(x) = 0$ και $Q(x) > 0$) ή\\
&($P(x) < 0$ και $Q(x) \in \mathbb{Z}$)\\
\hline
\end{tabular}
\end{center}

\begin{tcolorbox}[colback=purple!10!white, colframe=red!50!black, title=Ορισμός Γραφικής Παράστασης Συνάρτησης]
Έστω $f$ μια συνάρτηση με πεδίο ορισμου $A$ και $Oxy$ ενα συστημα συντενταγμένων στο επίπεδο. Το σύνολο των σημείων $M(x,y), x\in A$, λέγεται \textbf{γραφική παράσταση} της $f$ και συμβολίζεται με $C_f$.
\end{tcolorbox}

Από τον ορισμό της συνάρτησης και της γραφικής παράστασης, μπορούμε να αναπαραστή\-σουμε ποια είναι η συνάρτηση και ποια όχι. (Ο κυκλος δεν είναι συνάρτηση, καθώς ένα $x$ αντιστοιχεί σε παραπάνω απο ένα $y$)

\begin{center}
\begin{tikzpicture}[scale=1,>=stealth]

%%%%%% LEFT CHART (Exponential function) %%%%%%

  % Local scope
  \begin{scope}
    % Vertical grid lines only
    \foreach \x in {0.5,1,...,5.5} {
      \draw[very thin, color=gray!30] (\x,0) -- (\x,4);
    }

    % Axes
    \draw[->, thick] (-0.5,0) -- (6,0) node[right] {$x$};
    \draw[->, thick] (0,-0.5) -- (0,4) node[above] {$y$};

    % Origin label
    \node[below left] at (0,0) {$O$};

    % Dotted lines at domain endpoints
    \draw[dotted, thick] (1.2,0) -- (1.2,{exp(1.2/5)});
    \draw[dotted, thick] (4.8,0) -- (4.8,{exp(4.8/5)});

    % Plot y = e^{x/5}
    \draw[domain=1.2:4.8, smooth, variable=\x, blue, thick]
      plot ({\x},{exp(\x/5)});

    % Curly bracket under the x-axis
    \draw[decorate, decoration={brace, mirror, amplitude=6pt}, thick]
      (1.2, -0.3) -- (4.8, -0.3);

    % Label A under the curly bracket
    \node at (3, -0.7) {$A$};
  \end{scope}

%%%%%% RIGHT CHART (Circle) %%%%%%

  \begin{scope}[xshift=8cm] % shift to the right
    \draw[very thin, color=gray!30] (1.43,-0.5) -- (1.43,4);

    % Axes
    \draw[->, thick] (-0.5,0) -- (6,0) node[right] {$x$};
    \draw[->, thick] (0,-0.5) -- (0,4) node[above] {$y$};

    % Origin
    \node[below left] at (0,0) {$O$};

    % Circle of radius 2
    \draw[red, thick] (2.5,1.3) circle[radius=1.67];
    \node at (4.3, 2.6) {$C$};

    \tikzset{dot/.style={circle, fill=red, inner sep=1.5pt}}
    \node[dot] (a1) at (1.45,0) {};
    \node[dot] (a2) at (1.45,2.6) {};

  \end{scope}

\end{tikzpicture}
\end{center}

Έτσι από την γραφική παράσταση της $C_{f}$ μπορούμε να συμπαιράνουμε:
\begin{enumerate}
 \item Το πεδίο ορισμού της $f$ είναι το σύνολο $A$ των τετμημένων των σημείων της $C_{f}$.
 \item Το σύνολο τιμών της $f$ είναι το σύνολο $f(A)$ των τεταγμένων των σημείων της $C_{f}$.
 \item Η τιμή της $f$ στο $x_{0} \in A$ είναι η τεταγμένη του σημείου τομής της ευθείας $x=x_{0}$ και της $C_{f}$.
\end{enumerate}

\begin{center}
\begin{tikzpicture}[scale=1,>=stealth]

%%%%%% LEFT CHART (Exponential function) %%%%%%

  % Local scope
  \begin{scope}
    % Vertical grid lines only
    \foreach \x in {0.5,1,...,5.5} {
      \draw[very thin, color=gray!30] (\x,0) -- (\x,4);
    }

    % Axes
    \draw[->, thick] (-0.5,0) -- (6,0) node[right] {$x$};
    \draw[->, thick] (0,-0.5) -- (0,4) node[above] {$y$};

    % Origin label
    \node[below left] at (0,0) {$O$};

    % Dotted lines at domain endpoints
    \draw[dotted, thick] (1.2,0) -- (1.2,{exp(1.2/5)});
    \draw[dotted, thick] (4.8,0) -- (4.8,{exp(4.8/5)});

    % Plot y = e^{x/5}
    \draw[domain=1.2:4.8, smooth, variable=\x, blue, thick]
      plot ({\x},{exp(\x/5)});

    % Curly bracket under the x-axis
    \draw[decorate, decoration={brace, mirror, amplitude=6pt}, thick]
      (1.2, -0.3) -- (4.8, -0.3);

    % Label A under the curly bracket
    \node at (3, -0.7) {$A$};
  \end{scope}

%%%%%% RIGHT CHART (Circle) %%%%%%

  \begin{scope}[xshift=8cm] % shift to the right
    \draw[very thin, color=gray!30] (1.43,-0.5) -- (1.43,4);

    % Axes
    \draw[->, thick] (-0.5,0) -- (6,0) node[right] {$x$};
    \draw[->, thick] (0,-0.5) -- (0,4) node[above] {$y$};

    % Origin
    \node[below left] at (0,0) {$O$};

    % Circle of radius 2
    \draw[red, thick] (2.5,1.3) circle[radius=1.67];
    \node at (4.3, 2.6) {$C$};

    \tikzset{dot/.style={circle, fill=red, inner sep=1.5pt}}
    \node[dot] (a1) at (1.45,0) {};
    \node[dot] (a2) at (1.45,2.6) {};

  \end{scope}

\end{tikzpicture}
\end{center}

\end{document}
